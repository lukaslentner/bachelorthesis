\chapter[Klassische MCS am Beispiel des Ising Modells]{Klassische MCS\\\LARGE am Beispiel des Ising Modells}

Um die Grundlagen einer Monte Carlo Simulation (MCS) kennenzulernen, wurde im Vorfeld eine Anwendung zur Simulation des 2-dimensionalen Ising-Modells mit periodischer Randbedingung erstellt. Die gewonnene Erfahrung erwies sich f�r das Studium der quantenmechanischen MCS als �u�erst hilfreich.

\section{Methode}

\subsection{Das Ising-Modell}

F�r das klassische, ferromagnetische Ising-Modell ist der Hamiltonian

\begin{equation}
H_{\mathrm{Ising}}=-\sum_{\left\langle i,j\right\rangle}J_{ij}\cdot S_i^zS_j^z-B\mu\sum_{i=0}^{N-1}S_i^z
\label{equ:isinghamiltonian}
\end{equation}

zusammengesetzt aus einer $z$-Koppelung benachbarter Spins $\left\langle i,j\right\rangle$ (wird sp�ter �ber das Gittermodell definiert), die durch die Bindungsmatrix $\boldsymbol{J}$ gewichtet wird, und einer magnetischen Wechselwirkung, in die das externe Magnetfeld $\boldsymbol{B}=(0,0,B)^T$ und das Magnetische Moment $\boldsymbol{\mu}=(0,0,\mu)^T$ eingeht. F�r unser Beispiel setzten wir jedes $J_{ij}=1$ und betrachten die Anordnung ohne Magnetfeld -- da uns nur die $z$-Richtung interessiert setzen wir $S=S^z$. Der Hamiltonian erh�lt dann die vereinfachte Struktur,

\begin{equation}
H=-\sum_{\left\langle i,j\right\rangle}S_iS_j\ \mathrm{.}
\label{equ:beispielisinghamiltonian}
\end{equation}

\subsection{Kanonische �bergangswahrscheinlichkeiten}

Wegen der vorgegebenen Teilchenanzahl $N$ und Temperatur $T$ setzen wir f�r eine beliebige Gr��e $A$ den Mittelwert

\begin{equation}
\langle A\rangle =\sum_{\sigma\in\Omega}\frac{e^{-\beta E_\sigma}}{Z}\cdot A(\sigma )
\label{equ:kanonischermittelwert}
\end{equation}

kanonisch an, wobei

\begin{equation}
Z=\sum_{\sigma\in\Omega}e^{-\beta E_\sigma}
\label{equ:zustandssumme}
\end{equation}

die kanonische Zustandssumme, $\beta$ die reduzierte Temperatur $1/T$ und $E_\sigma$ die Energie eines gewissen mikroskopischen Zustandes $\sigma$ (Konfiguration) darstellt. Im Vergleich zur Gl. \ref{equ:mittelwert} sieht man, dass die Wahrscheinlichkeit eines Zustandes $\sigma$

\begin{equation}
p_\sigma=\frac{e^{-\beta E_\sigma}}{Z}
\label{equ:kanonischewahrscheinlichkeit}
\end{equation}

boltzmannverteilt ist. Diese Gewichte sind schwer zu berechnen, da der Zustandsraum in solch einem Spin-System exponentiell mit der Spinanzahl anw�chst ($\vert\Omega\vert\sim 2^N$). Machen wir allerdings vom Metropolis Algorithmus Gebrauch (siehe Gl. \ref{equ:metropolis}), ben�tigen wir die einzelnen Gewichte gar nicht, sondern k�nnen uns mit deren Verh�ltnissen, die dann die �bergangswahrscheinlichkeiten $\boldsymbol{W}$ darstellen, gen�gen,

\begin{equation}
W_{\nu\sigma}=\begin{cases}
e^{-\beta(E\sigma-E_\nu)} & E\sigma>E_\nu\\
1                         & E_\sigma\leq E_\nu\ \mathrm{.}
\end{cases}
\label{equ:kanonischermetropolis}
\end{equation}

\subsection{Messgr��en}

Folgende typischen, thermodynamischen Gr��en wollen wir in unserer Beispielanwendung messen:

\vspace{-2mm}
\begin{align}
\mathrm{Energie                      :}\quad & E=\left\langle H\right\rangle
\ \mathrm{,}\label{equ:energie}\\[6mm]
\mathrm{Magnetisierung\ pro\ Spin    :}\quad & M=\left\langle S_i\right\rangle
\ \mathrm{,}\label{equ:magnetisierung}\\[4mm]
\mathrm{Spezifische\ W"arme          :}\quad & C_V=\frac{1}{Nk\,T^2}\left(\left\langle H^2\right\rangle-\left\langle H\right\rangle^2\right)
\ \mathrm{,}\label{equ:spezwaerme}\\[3mm]
\mathrm{Magnetische\ Suszeptibilit"at:}\quad & \chi_m=\frac{N}{k\,T}\left(\left\langle S_i^2\right\rangle-\left\langle S_i\right\rangle^2\right)
\ \mathrm{.}\label{equ:suszeptibilitaet}
\end{align}

\section{Implementierung}

Hallo

\section{Ergebnisse}



relaxation time goes up for transition

aussicht cluster alg

finite size scaling critical exp

Vergleich mit Meanfield