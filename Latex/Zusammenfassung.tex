\newpage
\thispagestyle{empty}
\cleardoublepage
\chapter{Zusammenfassung}

Die {\bfseries Stochastic Series Expansion} stellt ein m�chtiges Werkzeug f�r das Sampling von Operatorstrings innerhalb eines quantenmechanischen L�sungsansatzes dar. Sie ist f�r gr��ere 2-dimensionale Systeme der schnellste Weg, um die typischen, thermodynamischen Gr��en zu messen und wird hierf�r an mehreren Instituten erfolgreich eingesetzt. Der Algorithmus ist relativ leicht zu implementieren und arbeitet �u�erst speichersparend.

Im Rahmen dieser Arbeit wurde ein Simulationsprogramm geschrieben, welches nicht nur ein blo�es SSE Modul enth�lt, sondern die M�glichkeit anbietet, die SSE-Daten f�r kleine Systeme mit ED {\bfseries Exakt Diagonalisation} zu �berpr�fen, als auch ein Zusatzmodul f�r das numerische L�sen des klassischen Ising-Modells enth�lt. Nach einem Baukasten Prinzip wurden Klassendateien f�r verschiedene Szenarien hinzugef�gt. So ist die Analyse der Daten in einzelne Pakete gekapselt, um ein H�chstma� an Flexibilit�t zu garantieren. Das Messen einer Gr��e kann also einfach ein- und ausgeschaltet werden. Weiterhin bietet die Anwendung verschiedene Geometriemodelle an, aus denen das gew�nschte gew�hlt werden kann (Periodische Gitter und Offene und Periodische Ketten). Nat�rlich ist es ohne Weiteres m�glich einfach weitere Features hinzuzuf�gen!

Die durchgef�hrten Messungen best�tig die Theorie und liefern sehr genaue Ergebnisse. �ber den Parameter "`measureCount"' kann die Anzahl der durchzuf�hrenden Messungen modifiziert werden. So ist eine Entscheidung zwischen ben�tigter Rechenzeit und Genauigkeit m�glich.