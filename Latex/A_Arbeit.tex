\documentclass[a4paper, twoside]{report}

\usepackage[latin1]{inputenc}
\usepackage[ngerman]{babel}
\usepackage{graphicx}
\usepackage{parskip}
\usepackage{tabularx}
\usepackage{url}
\usepackage{geometry}
\usepackage{hyperref}
\usepackage[TS1, T1]{fontenc}
\usepackage{lmodern, textcomp}
\usepackage{listings}
\usepackage[sumlimits, intlimits]{amsmath}
\usepackage{amsfonts}
\usepackage{amssymb}
\usepackage{nicefrac}
\usepackage{soul}
\usepackage[font=footnotesize]{caption}
\usepackage[font=footnotesize]{subfig}
\usepackage[numbers]{natbib}
\usepackage{blkarray}

\renewcommand{\date}{1. August 2011}

\newcommand{\email}[1]{\href{mailto:#1}{\nolinkurl{#1}}}

\renewcommand{\theenumi}{\alph{enumi}}
\renewcommand{\labelenumi}{\theenumi)}
\renewcommand{\theenumii}{\roman{enumii}}
\renewcommand{\labelenumii}{\theenumii)}

\newcommand{\cppcodefile}[2][]{\lstinputlisting[basicstyle=\ttfamily\footnotesize,breaklines=true,showstringspaces=false,numberstyle=\ttfamily\tiny,language=c++,numbers=left,numbersep=8mm,xleftmargin=15mm,framextopmargin=1mm,framexleftmargin=5mm,framexbottommargin=1mm,captionpos=b,frame=trbl,caption={#2},#1]{../Code/Source/#2}}

\DeclareMathOperator{\tr}{tr}
\DeclareMathOperator{\var}{var}
\DeclareMathOperator{\even}{even}
\DeclareMathOperator{\odd}{odd}

\renewcommand*\arraystretch{1.5}

\hypersetup{
  pdfauthor   = {Lukas B. Lentner},
  pdftitle    = {Stochastic Series Expansion: Quanten Monte-Carlo Simulation des Heisenberg Modells},
  pdfsubject  = {Computergest�tzte numerische Physik},
  pdfkeywords = {QMS, SSE, Heisenberg Modell, Monte-Carlo, Simulation Stochastic Series Expansion, Thermodynamik},
  pdfcreator  = {LaTeX with hyperref package},
  pdfproducer = {dvips + ps2pdf}
}

% \chapter{Example}
% \section{Moddeling}
% \subsection{The Model}
% \subsubsection{Some Part}

% \begin{cppcode}[caption={Titel oder so2},label=code:Sourcecode2]
% \end{lstlisting}

% \cppcodefile[caption={Titel oder so2},label=code:Sourcecode2]{file.cpp}

% \begin{figure}[h]
%  \centering
%  \includegraphics[width=0.30\textwidth]{Images/LMU-Siegel}
%  \caption{Titel oder so2}
%  \label{fig:Siegel2}
% \end{figure}

% \begin{table}
%  \caption{Titel oder so2}
%  \begin{tabular}{| r | r || c | c | c |}
%   Hallo&Ich&Bin&der&ich\\
%   Bin&Da&auch&du&da
%  \end{tabular}
%  \label{tbl:Test2}
% \end{table}

% \footnote{Footnotetext}

% \cite{tag}

% \ref{equ:Bragg}

%\includeonly{A_Titelseite, A_Inhaltsverzeichnis, Theorie, Klassisch, A_Literaturverzeichnis}

\begin{document}

\begin{titlepage}

\newgeometry{left=40mm,right=40mm,top=60mm,bottom=45mm} 
\centering
\sodef\so{}{.14em}{.4em plus.1em minus .1em}{.4em plus.1em minus .1em}
\textsc{\Large Implementierung der}\\[7mm]

\hrule\vspace{2mm}
 
\textsc{\LARGE\so{Stochastic Series Expansion}}\\[7mm]

\hrule\vspace{2mm}

\textsc{\Large f�r Spin -- $\nicefrac{1}{2}$ Heisenberg Systeme}\\[20mm]

\date\vfill

\includegraphics[height=50mm]{Images/LMU-Siegel}\\[35mm]

{\Large Lukas B. Lentner}

\end{titlepage}

\newpage
\thispagestyle{empty}
\cleardoublepage
\chapter*{Vorwort}
\addcontentsline{toc}{chapter}{Vorwort}

\begingroup
\leftskip=10mm
\textit{�Man versteht etwas nicht wirklich,\\
wenn man nicht versucht, es zu implementieren.�}\\[2mm]
--- von \textsc{Donald Ervin Knuth} \cite{Knuth}
\vspace{5mm}
\par
\endgroup

TODO: Dank an Prof. Dr. Ulrich Schollw�ck und Dr. Fabian Heidrich-Meisner
TODO: Dank an Mutter, Vater und Bruder
TODO: Dank an Freundin

Generelles TODO:

-----------------------------------------------

Noch machen: Einleitung, Vorwort und Zusammenfassung

F�r Einleitung sind (Etwas Interessantes, Literatur�bersicht und SectionOutline wichtig), Beschreibe andere Methoden: Analytisch Besel, DMRG, Exakt Diagonalisation auch begrenzt, da SSE durch Sign Problem begrentzt, Wei�sche Bezirke und CorrLength erw�hnen einf�hren (Verhalten beim kritischen Punkt, auch grundzustand und sog. Thermisches Chaos)

�berpr�fe Formalit�ten:
\begin{itemize}
	\item Cite im Verzeichnis
	\item F�r jedes Float ordentliche Untershcrift und Verzeichniseintrag
	\item F�r jedes Float die Quelle angeben
	\item Noch andere Literatur wie im Ref und Nolting ...
	\item �berpr�fe: �berschriften, Rechtschreibung, Aufbau, Mehrfacherw�hung
\end{itemize}

Bewertung:
\begin{itemize}
	\item Beschreibe Ergebnisse
	\item Diskussion der Ergebnisse
	\item Stelle heraus was du gemacht hast!
	\item Zeige, dass du es verstanden hast
\end{itemize}


---------------------------------------------

\vspace{6mm}
\begin{tabularx}{\textwidth}{@{}X@{}@{}r@{}}
Lukas B. Lentner&M�nchen, \date\\
\email{kontakt@lukaslentner.de}
\end{tabularx}

\let\contentsnamePARENT\contentsname
\renewcommand{\contentsname}{\pdfbookmark{\contentsnamePARENT}{toc}\contentsnamePARENT}
\tableofcontents

%=======================================

\chapter{Einleitung}

Bezug auf \cite{Diplom}, \cite{Sandvik}, \cite{Buch}

\chapter[Theorie der Monte Carlo Simulation]{Theorie der\\ Monte Carlo Simulation}

\section{Geschichte}

Basierend auf den Ideen von Enrico Fermi (um 1935) verwendete zum ersten Mal Stanislaw Ulam und John von Neumann um 1945 das Prinzip der Monte Carlo Methode w�hrend ihrer Arbeit am Los Alamos Scientific Laboratory.

Der von Nicholas Metropolis gew�hlte Name bezieht sich auf die Spielbank Monte Carlo, die im gleichnamigen Stadtteil des Stadtstaates Monaco liegt. Anlass hierf�r soll Ulams Onkel gegeben haben, der sich mehrmals von Verwandten Geld zum spielen leihen wollte \cite{mchistory}.

Heute findet die Methode zahlreiche Anwendungen in der Statistischen Physik, Numerik und Optimierung.

\section{Ziel}

Die Idee der Monte Carlo Simulation (MCS) l�sst sich beschreiben als ein gewichteter Weg durch einen $n$-dimensionalen Zustandsraum $\Omega$. Hierbei interessiert man sich speziell f�r den statistischen Mittelwert einer Gr��e $A$,

\begin{equation}
\langle A\rangle =\sum_{\sigma\in\Omega}p_\sigma\cdot A(\sigma )\ \mathrm{.}
\label{equ:mittelwert}
\end{equation}

$p_\sigma$ steht hier f�r die Wahrscheinlichkeit des Zustandes $\sigma$ und $A(\sigma )$ ist der Wert der Gr��e $A$ bei diesem Zustand. F�r kontinuierliche F�lle ersetzt man die Summe durch ein Integral.

\section{Markov-Kette}

Oft ist es nicht m�glich, die oben angegebene Summe auszuwerten (z.B. wenn $\Omega$ sehr gro� ist). In diesem Fall kann der Zustandsraum quasidicht durch eine Markov-Kette von $M$ Zust�nden $\sigma_0, \sigma_1, \ldots \sigma_{M-1}$ abgelaufen werden. Die H�ufigkeit eines Zustandes $\sigma$ in der Kette soll im Grenzfall $M\rightarrow\infty$ genau der Wahrscheinlichkeit des Zustandes $p_\sigma$ entsprechen (Importance Sampling). Der Mitterwert kann sodann erheblich leichter nach dem Gesetz f�r gro�e Zahlen durch das arithmetische Mittel �ber die Kette, also

\begin{equation}
\langle A\rangle\approx\overline{A}=\frac{1}{M}\sum_{m=0}^{M-1}A(\sigma_m)\ \mathrm{,}
\label{equ:markov_mittelwert}
\end{equation}

gen�hert werden.

Eine Markov-Kette beginnt mit einem beliebigen Anfangszustand $\sigma_0$. Von diesem aus werden mit einer �bergangswahrscheinlichkeit $W_{\sigma_0\sigma_1}$ Spr�nge im Zustandsraum ausgef�hrt (MC-Schritte), welche die neuen Kettenglieder $\sigma_2\ldots$ definieren. Damit die Markov-Kette zur gew�nschten Wahrscheinlichkeitsverteilung f�hrt, muss bei der Bildung von $\boldsymbol{W}$ auf die zwei folgenden Bedingungen geachtet werden:

\begin{enumerate}
\item Die Bildung der Kette muss {\bfseries ergodisch} sein. D.h. sie muss theoretisch alle Zust�nde enthalten k�nnen, was sie in der Praxis nat�rlich nicht tut, da wir $M\ll \left|\Omega\right|$ w�hlen.
\item Die �bergangswahrscheinlichkeiten $\boldsymbol{W}$ m�ssen insofern im {\bfseries Gleichgewicht} sein, als dass
\end{enumerate}

\begin{equation}
\sum_{\sigma\in\Omega}p_\sigma\cdot W_{\sigma\nu}=p_\nu\ \mathrm{.}
\label{equ:balanced}
\end{equation}

Eine deutlich st�rkere Bedingung als b) stellt {\itshape Detailed Balance} (dt. detailiertes Gleichgewicht) dar,

\begin{equation}
p_\sigma\cdot W_{\sigma\nu}=p_\nu\cdot W_{\nu\sigma}\ \mathrm{.}
\label{equ:detailedBalance}
\end{equation}

In Worten besagt sie, dass ein Sprung von einem Markov-Kettenglied zum Nachbar genauso wahrscheinlich ist, wie andersherum. Die Kette besitzt also keine ausgezeichnete Richtung.

Gleichung \ref{equ:detailedBalance} erf�llt automatisch Gl. \ref{equ:balanced}, da

\begin{equation}
\sum_{\sigma\in\Omega}p_\sigma\cdot W_{\sigma\nu}=\sum_{\sigma\in\Omega}p_\nu\cdot W_{\nu\sigma}=p_\nu\cdot\sum_{\sigma\in\Omega}W_{\nu\sigma}=p_\nu\ \mathrm{.}
\label{equ:detailedBalanceIsBalanced}
\end{equation}

Hierbei verwendet man im letzten Schritt, dass der Zustand $\nu$ in jedem Fall in irgendeinen n�chsten Zustand $\sigma$ �bergeht. Diese zun�chst starke Einschr�nkung wird h�ufig verwendet, um der Bedingung b) zu gen�gen. Sp�ter werden wir sehen, dass sie in unserem Fall auch die Berechnung von $\boldsymbol{W}$ deutlich vereinfacht.

\section{Metropolis Algorithmus}

Ein m�glicher Algorithmus zur Bestimmung der �bergangswahrscheinlichkeiten $\boldsymbol{W}$ wurde 1953 von Nicholas Metropolis et al. vorgestellt \cite{metropolis},

\begin{equation}
W_{\nu\sigma}=\begin{cases}
p_\sigma/p_\nu & p_\sigma<p_\nu\\
1              & p_\sigma\geq p_\nu\ \mathrm{.}
\end{cases}
\label{equ:metropolis}
\end{equation}

Es kann leicht gezeigt werden, dass der Vorschlag die {\itshape Detailed Balance} (Gl. \ref{equ:detailedBalance}) erf�llt. Ein weiterer Algorithmus ist nach Roj J. Glauber benannt ({\itshape Glauber dynamics}) \cite{glauber}.

\section{Thermalisierung}

Nachdem als Anfangszustand der Markov-Kette ein beliebig ausgew�hlter Zustand verwendet wird, ist es ziemlich unwahrscheinlich, dass dieser Zustand ein hohes Wahrscheinlichkeitsgewicht $P_\sigma$ besitzt. Es wird sich also nicht um einen Zustand im Gleichgewicht handeln. Aus diesem Grund sollte vor der eigentlichen Messung eine gen�gend gro�e Anzahl von Thermalisierungsschritten (MC-Schritte) durchgef�hrt werden.

In der Praxis werden entweder Erfahrungswerte verwendet, die eine konstante, meist zu gro�e Schrittanzahl erfordern oder die Daten werden vollst�ndig gespeichert und in der Auswertung sortiert. Im Nachhinein kann �ber tats�chliche Wahrscheinlichkeitsverteilung auf die Thermalisierungsphase geschlossen werden. Diese Daten werden dann f�r die anschlie�ende Analyse nicht verwendet.

\section{Autokorrelationsfunktion und Fehlerberechnung}

Alle Messwerte der Gr��e $A$ m�ssen nach der Termalisierung in der Auswertung statistisch interpretiert werden. Dabei ist zu beachten, dass die Daten von aufeinanderfolgenden Zust�nden statistisch abh�ngig sind. Wie viele MC-Schritte zwischen zwei Messungen notwendig sind, um unabh�ngige Werte zu erhalten, gibt die Autokorrelationszeit $\tau_A$ an (Im weiteren ist mit "`Zeit"' immer die Simulationszeit gemessen in MC-Schritten gemeint). Zur Berechnung derselben wird die Autokorrelationsfunktion

\begin{equation}
\Theta_A(t)=\frac{\langle A(\sigma_{i+t})\cdot A(\sigma_i)\rangle-\langle A\rangle^2}{\langle A^2\rangle-\langle A\rangle^2}
\label{equ:autokorrelationsfunktion}
\end{equation}

betrachtet. Hierbei l�uft die Mittelwertbildung mit der Variable $i$ �ber die gesamte ausgewertete Simulationszeit $\widetilde{M}$ (Die gesamte Simulationszeit betr�gt $M$). Sie ist in solch einer Weise normiert, dass $\Theta_A(\sigma_0)=1$ und $\Theta_A(\sigma_{t\rightarrow\infty})=0$. Die Autokorrelationsfunktion h�ngt sodann mit der Autokorrelationszeit negativ exponentiell zusammen,

\begin{equation}
\Theta_A(t)\sim e^{-t/\tau_A}\ \mathrm{.}
\label{equ:autokorrelationszeit}
\end{equation}

Nachdem $\tau_A$ auf diese Art ermittelt wurde, k�nnen die Messwerte in Gruppen mit der L�nge $3\cdot\tau_A$ geb�ndelt und unabh�ngige Gruppenmittelwerte

\begin{equation}
\overline{A}_b=\frac{1}{3\tau_A}\sum_{i=0}^{3\tau_A - 1}A(\sigma_{b\cdot\tau_a+i})
\label{equ:gruppenmittelwert}
\end{equation}

berechnet werden, wobei $b$ hier der null-basierte Gruppen-Index ist und die Anzahl der Gruppen

\begin{equation}
B=\left\lfloor\frac{\widetilde{M}}{3\tau_A}\right\rfloor\ \mathrm{.}
\label{equ:gruppenanzahl}
\end{equation}

Nach dem zentralen Grenzwert Satz folgen diese Gruppenmittelwerte sodann einer Gau�-Vertei\-lung, welche den exakten Wert

\begin{equation}
\overline{A}=\frac{1}{B}\sum_{b=0}^{B-1}\overline{A}_b
\label{equ:exact}
\end{equation}

in der Mitte h�lt. Als Fehler kann eine Standardabweichung, also 

\begin{equation}
\sigma_A=\sqrt{\frac{1}{B(B-1)}\sum_{b=0}^{B-1}(\overline{A}_b-\overline{A})^2}
\label{equ:error}
\end{equation}

angegeben werden.

\newpage
\thispagestyle{empty}
\cleardoublepage
\chapter[Klassische MCS am Beispiel des Ising-Modells]{Klassische MCS\\\LARGE am Beispiel des Ising-Modells}
\label{sec:Ising}

Um die Grundlagen der Monte-Carlo Simulation (MCS) kennenzulernen, betrachten wir zuerst die Simulation des klassischen, 2-dimensionalen Ising-Modells mit periodischer Randbedingung. Als Messgr��en w�hlen wir die typischen thermodynamischen Gr��en: Den Mittelwert der Energie, W�rmekapazit�t, Magnetisierung und magnetischen Suszeptibilit�t. Au�erdem betrachten wir die absolute Magnetisierung und die absolute magnetische Suszeptibilit�t, also den Mittelwert des Absolutbetrags der Spin-Summe und dessen Varianz (weil sich ohne �u�eres Magnetfeld die Magnetisierung immer auf 0 mittelt). All diese Gr��en werden pro Spin gemessen.

Um ein Gegen�berstellen zu erleichtern, hat dieses und das n�chste Kapitel eine analoge Struktur: Im ersten Abschnitt kn�pfen wir an Kapitel \ref{sec:Theorie} an, d.h. wir werden unseren Zustandsraum $\Omega$ und die �bergangswahrscheinlichkeiten $\boldsymbol{W}$ f�r dieses Szenario definieren. Danach betrachten wir die Implementierung der erstellten Anwendung detailiert. Im letzten Abschnitt werden die Ergebnisse verschiedener Simulationen vorgestellt und diskutiert.

\section{Methode}

\subsection{Das Ising-Modell}

F�r das klassische, ferromagnetische Ising-Modell ist der Hamiltonian

\begin{equation}
H_{\mathrm{Ising}}=-\sum_{\left\langle i,j\right\rangle}J_{ij}\cdot S_i^zS_j^z-h\sum_{i=0}^{N-1}\mu_{i}\cdot S_i^z
\label{eq:IsingHamiltonian}
\end{equation}

zusammengesetzt aus einer magnetischen $z$-Koppelung benachbarter Spins $\left\langle i,j\right\rangle$, die durch die Bindungsmatrix $\boldsymbol{J}$ gewichtet wird, und der Wechselwirkung eines externen Magnetfelds $\boldsymbol{h}=(0,0,h)^T$ mit den magnetischen Momenten $\boldsymbol{\mu}=(0,0,\mu)^T$. F�r unser Beispiel setzten wir alle $J_{ij}=1$ sowie $\mu_i=1$ (Homogenit�t) und betrachten die Anordnung ohne Magnetfeld ($h=0$) -- da uns nur die $z$-Richtung interessiert, setzen wir $S=S^z\in\{-1;1\}$. Der Hamiltonian erh�lt dann die vereinfachte Struktur:

\begin{equation}
H=-\sum_{\left\langle i,j\right\rangle}S_iS_j
\label{eq:BeispielIsingHamiltonian}
\end{equation}

\subsection{Sampling}
\label{sec:KlassischesSampling}

Wegen der vorgegebenen Teilchenanzahl $N$ und Temperatur $T$ k�nnen wir f�r eine beliebige Gr��e $A$ den Mittelwert

\begin{equation}
\langle A\rangle =\sum_{\sigma\in\Omega}\frac{e^{-\beta E_\sigma}}{Z}\cdot A(\sigma )
\label{eq:KanonischerMittelwert}
\end{equation}

als kanonisches Ensamble ansetzen, wobei

\begin{equation}
Z=\sum_{\sigma\in\Omega}e^{-\beta E_\sigma}
\label{eq:Zustandssumme}
\end{equation}

die kanonische Zustandssumme, $\beta$ die reduzierte Temperatur $1/T$ (wir setzten $k_B=1$) und $E_\sigma$ die Energie eines gewissen mikroskopischen Zustandes $\sigma$ (Konfiguration) darstellt. Analog zum Abschnitt \ref{sec:Metropolis} wenden wir nun die {\bfseries Monte-Carlo Methode} auf diese Konfigurationen $\in\{1,2\}^N$ an. Im Vergleich zur Gl. \ref{eq:Mittelwert} sieht man hierbei, dass die Wahrscheinlichkeit eines Zustandes Boltzmann-verteilt ist:

\begin{equation}
p_\sigma=\frac{e^{-\beta E_\sigma}}{Z}
\label{eq:KanonischeWahrscheinlichkeit}
\end{equation}

Die Gewichte (speziell die Zustandssumme) sind allerdings schwer zu berechnen, da der Zustandsraum in solch einem Spin-System exponentiell mit der Spinanzahl anw�chst ($\vert\Omega\vert\sim 2^N$) und eine numerische Berechnung von $Z$ f�r gro�e Systeme oft nicht mehr m�glich ist. F�r den {\bfseries Metropolis Algorithmus} (siehe Gl. \ref{eq:Metropolis}), ben�tigen wir allerdings diese einzelnen Gewichte gar nicht, sondern k�nnen uns mit deren Verh�ltnissen, die dann die �bergangswahrscheinlichkeiten $\boldsymbol{W}$ darstellen, begn�gen:

\begin{equation}
W_{\nu\sigma}=\begin{cases}
e^{-\beta(E_\sigma-E_\nu)} & E_\sigma>E_\nu\\
1                          & E_\sigma\leq E_\nu
\end{cases}
\label{eq:KanonischerMetropolis}
\end{equation}

\section{Implementierung}
\label{sec:KlassischeImplementierung}

Die Anwendung orientiert sich an \cite{Sandvik}. Sie gliedert sich grob in die Initialisierung des Systems, die Simulation des Modells sowie die Analyse der Messdaten. Um die gew�nschten Gr��en abh�ngig von der Temperatur betrachten zu k�nnen, f�hren wir das Programm f�r mehrere Temperaturen aus.

\subsection{Initialisierung}

Generell m�ssen zuerst folgende Parameter festgelegt werden:

\begin{itemize}
\item Anzahl der Spins $N$,
\item Anzahl der Messungen $R_1$ und
\item Temperatur des Systems $T$.
\end{itemize}

F�r den Status der Spins legen wir ein boolsches Array der L�nge $N$ an und initialisieren es mit zuf�lligen Werten (Anfangszustand). Da sich alle Messgr��en von der Energie und der Spin-Summe ($\approx$ Magnetisierung, siehe Abschnitt \ref{sec:KlassischeErgebnisse}) 

\begin{equation}
S=\sum_{i=0}^{N-1}S_i
\label{eq:SpinSumme}
\end{equation}

ableiten lassen, speichern wir immer deren aktuelle Werte ab. Wie wir sp�ter sehen werden, k�nnen wir beide Werte in jedem MC-Schritt direkt angepassen (Update) und m�ssen diese nicht jedes Mal erneut berechnen (zu Beginn ist dies aber nat�rlich vonn�ten).

\subsection{Simulation}

Um eine Markov-Kette der L�nge $R$ zu sampeln, verwenden wir eine Schleife, die jeweils einen MC-Schritt durchf�hrt. Ab $R_1$ Durchl�ufen (Thermalisierung, siehe Abschnitt \ref{sec:Thermalisierung}) legen wir jedes Mal zus�tzlich die aktuelle Energie, die Magnetisierung ($M=S/N$) und die absolute Magnetisierung ($M'=\vert S\vert/N$) in einem geeigneten Array ab.

\paragraph{Monte-Carlo Schritt}

Wir erzeugen je das n�chste Markov-Kettenglied, indem wir versuchen, jeden Spin des Systems umzudrehen (engl. flip). Das Umdrehen wird jeweils gestattet, wenn eine Zufallszahl zwischen 0 und 1 kleiner ist als das Boltzmanngewicht

\begin{equation}
e^{-\beta\Delta E}\ \mathrm{,}
\label{eq:Gewicht}
\end{equation}

wobei $\Delta E$ der Energieunterschied zwischen der neuen, m�glichen Konfiguration und der aktuellen ist. Damit decken wir bereits die Gl. \ref{eq:KanonischerMetropolis} voll ab, da die Zufallszahl im zweiten Fall ($\Delta E<0\Rightarrow W_{\nu\sigma}=1$) auf jeden Fall kleiner ist als das Boltzmanngewicht.

\paragraph{Updates}

Wird das Umdrehen eines Spins erlaubt, modifizieren wir das Spin-Array und addieren zur aktuellen Energie und Spin-Summe den berechneten Unterschied $\Delta E$ und $\Delta S$:

\begin{itemize}
\item Zu $\Delta E$ tragen nur die Koppelungen zwischen dem Spin, den wir umdrehen wollen, und dessen Nachbaren bei. Diese sind im 2-dimensionalen Gitter die vier Spins �ber, unter sowie links und rechts von ihm.
\item $\Delta S$ ergibt sich einfach aus dem alten Status des Spins ($\pm2$).
\end{itemize}

\subsection{Analyse}

Die Mittelwerte folgender Gr��en wollen wir berechnen {\bfseries(immer pro Spin)}:

\begin{align}\
\mathrm{Energie:}\quad & \left\langle\frac{E}{N}\right\rangle=\frac{-\partial_\beta\ln Z}{N}=\left\langle \frac{H}{N}\right\rangle\ \mathrm{,}\label{eq:Energie}\\[2mm]
\mathrm{W"armekapazit"at:}\quad & \left\langle\frac{C}{N}\right\rangle=\frac{\partial_T H}{N}=\frac{N}{T^2}\left(\left\langle\left(\frac{H}{N}\right)^2\right\rangle-\left\langle\frac{H}{N}\right\rangle^2\right)\ \mathrm{,}\label{eq:Waermekapazitaet}\\[2mm]
\mathrm{Magnetisierung:}\quad & \left\langle\frac{M}{N}\right\rangle=\frac{T\partial_B\ln Z}{N}=\left\langle\frac{S_i}{N}\right\rangle\ \mathrm{,}\label{eq:Magnetisierung}\\[2mm]
\mathrm{magnetische\ Suszeptibilit"at:}\quad & \left\langle\frac{\chi}{N}\right\rangle=\frac{\partial_B M}{N}=\frac{N}{T}\left(\left\langle\left(\frac{S_i}{N}\right)^2\right\rangle-\left\langle\frac{S_i}{N}\right\rangle^2\right)\ \mathrm{,}\label{eq:Suszeptibilitaet}\\[2mm]
\mathrm{abs.\ Magnetisierung:}\quad & \left\langle\frac{M'}{N}\right\rangle=\left\langle\left|\frac{S_i}{N}\right|\right\rangle\ \mathrm{,}\label{eq:AbsoluteMagnetisierung}\\[2mm]
\mathrm{abs.\ mag.\ Suszeptibilit"at:}\quad & \left\langle\frac{\chi'}{N}\right\rangle=\frac{N}{T}\left(\left\langle\left|\frac{S_i}{N}\right|^2\right\rangle-\left\langle\left|\frac{S_i}{N}\right|\right\rangle^2\right)\ \mathrm{.}\label{eq:AbsoluteSuszeptibilitaet}
\end{align}

F�r jede dieser Gr��en werden -- wie in Abschnitt \ref{sec:Autokorrelation} ausgef�hrt -- zuerst die Autokorrelationszeit berechnet und anschlie�end die Messdaten gruppiert und schlie�lich gemittelt.

\subsection{Quellcode}

Der vom Author geschriebene C++ Quellcode ist im Anhang \ref{sec:code} zu finden. Folgende Dateien sind f�r diese, klassische Simulation relevant:

\begin{itemize}
\item\ref{code:SIM}: Hauptprogramm SIM
\item\ref{code:AbstractLattice}: Abstrakte Gitterklasse
\item\ref{code:Periodic2DLattice}: 2D Gitter mit periodischen Randbedingungen
\item\ref{code:AbstractAlgorithm}: Abstrakte Algorithmusklasse
\item\ref{code:ISINGAlgorithm}: Ising Algorithmus
\item\ref{code:AbstractAnalyzer}: Abstrakte Analyseklasse
\item\ref{code:IsingEnergyAnalyzer}: Analyse f�r die Energie (Ising)
\item\ref{code:IsingHeatCapacityAnalyzer}: Analyse f�r die W�rmekapazit�t (Ising)
\item\ref{code:IsingMagnetisationAnalyzer}: Analyse f�r die Magnetisierung (Ising)
\item\ref{code:IsingSusceptibilityAnalyzer}: Analyse f�r die magnetische Suszeptibilit�t (Ising)
\item\ref{code:IsingAbsoluteMagnetisationAnalyzer}: Analyse f�r die absolute Magnetisierung (Ising)
\item\ref{code:IsingAbsoluteSusceptibilityAnalyzer}: Analyse f�r die absolute magnetische Suszeptibilit�t (Ising)
\end{itemize}

\section{Ergebnisse und Diskussion}
\label{sec:KlassischeErgebnisse}

Bei der graphischen Aufbereitung wurde der �bersichtlichkeit wegen auf eine Angabe des Fehlers verzichtet (im Hauptteil werden wir sie gesondert darstellen). F�r die Mittelwerte der Grundgr��en $E$, $M$ und $M'$ sind diese kleiner als graphisch darstellbar. Mittelwerte weiterf�hrender Gr��en $C$, $\chi$ und $\chi'$ besitzen dagegen �blicherweise einen signifikanten Fehler in der N�he des Phasen�bergangs \cite{Nolting}

\begin{equation}
T_c=\frac{2}{\ln(1+\sqrt{2})}\approx 2.269185\ \mathrm{,}
\label{eq:IsingTC}
\end{equation}

ansonsten gilt dasselbe wie bei den Grundgr��en.

\subsection{Mittelwert der Energie und W�rmekapazit�t}
\label{sec:IsingEnergie}

\begin{figure}[bh]
  \centering
  \subfloat[{\bfseries Mittelwerte der Energie}]{
    \label{fig:KMCSEnergie}
    \includegraphics[width=0.48\textwidth]{Diagramme/KMCS/Energie-Temperatur} 
  }
  \subfloat[{\bfseries Mittelwerte der W�rmekapazit�t}]{
    \label{fig:KMCSWaermekapazitaet}
    \includegraphics[width=0.48\textwidth]{Diagramme/KMCS/Waermekapazitaet-Temperatur} 
  }
  \caption[Mittelwerte der Energie und W�rmekapazit�t; {\itshape Quelle:} Eigenwerk]{Mittelwerte der Energie und W�rmekapazit�t f�r verschieden gro�e Gitter mit periodischen Randbedingungen bei 10000 Messpunkten pro Temperaturpunkt; {\itshape Quelle:} Eigenwerk}
  \label{fig:KMCSEnergieWaermekapazitaet}
\end{figure}

Der Mittelwert der {\bfseries Energie} (Gl. \ref{eq:Energie}) in Abb. \ref{fig:KMCSEnergie} verl�uft erwartungsgem�� von -2 nach 0: F�r kleine Temperaturen stehen alle Spins in die gleiche Richtung ({\itshape Grundzustand}), da die Wahrscheinlichkeit eines Flips (Gl. \ref{eq:Gewicht}) eines einzelnen Spins gegen alle anderen verschwindend gering ist. Weil in einem 2-dimensionalen Gitter mit periodischen Randbedingungen die Anzahl der Koppelungen $N_b=2N$ ist, hat der Mittelwert hier den Wert -2. Je h�her jedoch die Temperatur steigt, je geringer wird der Einfluss des Boltzmann-Gewichts und f�hrt letztendlich zu einer Gleichverteilung der Spins, die f�r $T\rightarrow\infty$ $\left\langle E/N\right\rangle\rightarrow0$ liefert ({\itshape thermisches Chaos}).

Am Mittelwert der {\bfseries W�rmekapazit�t} (Gl. \ref{eq:Waermekapazitaet}) in Abb. \ref{fig:KMCSWaermekapazitaet}, also der Ableitung der mittleren Energie nach $T$, erkennt man deutlich den Phasen�bergang, der sich durch den h�chsten Wert f�r die Steigung der mittleren Energie erkenntlich macht. Dies erkl�rt sich durch die Bildung von gleichartig ausgerichteten (korrelierten) Spin-Clustern (Wei�sche Bezirke) in der Gr��enordnung der Korrelationsl�nge $\xi$ \cite{Nolting}. Da die Gr��e au�erdem die Varianz der Energie darstellt, erkl�rt sich der Verlauf ebenfalls aus der starken Reaktion der Energie auf geringste Temperatur�nderungen; hingegen sind Grundzustand und thermisches Chaos weitgehend "`stabil"'.

Ein Vergleich {\bfseries verschiedener Systemgr��en} zeigt eine Verschiebung des Peaks der mittleren W�rmekapazit�t als auch dessen Anwachsen, w�hrend abseits des Phasen�bergangs kein Unterschied festzustellen ist. Der Grund hierf�r kann wieder mit den Wei�schen Bezirken plausibel gemacht werden (siehe Seite 33 in \cite{Sandvik}):

\begin{itemize}
	\item $T\ll T_c$: Nahe am Grundzustand erwarten wir unabh�ngig von der Systemgr��e einen $\infty$-gr��en Bezirk mit vereinzelten St�rungen ($\xi=\infty$).
	\item $T\gg T_c$: Im thermischen Chaos ($\xi$ klein) von gr��tenteils dekorrelierten Einzelspins spielt die makroskopische Systemgr��e $N$ keine Rolle.
	\item $T\approx T_c$: Nahe dem Phasen�bergang nimmt jeder Bezirk einen makroskopischen Anteil des Systems ein, sodass sich das Verhalten bereits bei kleinen �nderungen massiv �ndert.
\end{itemize}

Um schlie�lich die reale �bergangstemperatur $T_c$ f�r $N\rightarrow\infty$ (thermodynamischer Limes) zu finden und die kritischen Exponenten bestimmen zu k�nnen, m�ssen wir einen polynominalen Fit �ber mehrere Systemgr��en hinweg verwenden (Finite Size Scaling) \cite{Buch}. F�r die obige Simulation ergab sich:

\begin{table}[htb]
	\centering
  \begin{tabular}{|l|l|l|l|}
    \hline
  	         & Eigene Werte & Exakter Wert & Wert durch Molekularfeldn�herung \\
    \hline
    $T_c$    & 2.26228      & 2.269185     & {\itshape Kein Wert} \\
    $\nu$    & 1            & 1            & 0.5 \\
    $\gamma$ & 1.75958      & 1.75         & 1   \\
    $\beta$  & 0.125        & 0.125        & 0.5 \\
    \hline
	\end{tabular}
	\caption{�bergangstemperatur und kritische Exponenten}
	\label{tab:kritischeexponenten}
\end{table}

\subsection{Autokorrelationszeit der Energie}

Wie wir sehen, nimmt auch die {\bfseries Autokorrelationszeit der Energie} $\tau_E$ (Gl. \ref{eq:Autokorrelationszeit}) in Abb. \ref{fig:KMCSAutokorrelationszeitEnergie} um den Phasen�bergang herum stark zu. Das bedeutet, dass die Konfigurationen �ber mehrere MC-Schritte hinweg korrelieren bzw. statistisch abh�ngig sind. Begr�ndet liegt dies in der Tatsache, dass die Korrelationsl�nge $\xi$ -- wie schon mehrfach erw�hnt -- am Phasen�bergang divergiert und die makroskopische Propagation von Information durch das System viel Zeit ben�tigt \cite{Buch}. $\tau_E$ ist neben der Temperatur auch vom System insbesondere dessen Gr��e abh�ngig.

\begin{figure}
  \centering
  \includegraphics[width=0.48\textwidth]{Diagramme/KMCS/AutokorrelationszeitEnergie-Temperatur} 
  \caption[Autokorrelationszeiten der Energie; {\itshape Quelle:} Eigenwerk]{{\bfseries Autokorrelationszeiten der Energie} f�r verschieden gro�e Gitter mit periodischen Randbedingungen bei 10000 Messpunkten pro Temperaturpunkt; {\itshape Quelle:} Eigenwerk}
  \label{fig:KMCSAutokorrelationszeitEnergie}
\end{figure}

\subsection{Mittelwert der Magnetisierung und magnetischen Suszeptibilit�t}

\begin{figure}[bh]
  \centering
  \subfloat[{\bfseries Mittelwerte der Magnetisierung}]{
    \label{fig:KMCSMagnetisierung}
    \includegraphics[width=0.48\textwidth]{Diagramme/KMCS/Magnetisierung-Temperatur} 
  }
  \subfloat[{\bfseries Mittelwerte der mag. Suszeptibilit�t}]{
    \label{fig:KMCSSuszeptibilitaet}
    \includegraphics[width=0.48\textwidth]{Diagramme/KMCS/MagnetischeSuszeptibilitaet-Temperatur} 
  }
  \caption[Mittlere Magnetisierung und magnetische Suszeptibilit�t; {\itshape Quelle:} Eigenwerk]{Mittlere Magnetisierung und magnetischen Suszeptibilit�t f�r verschieden gro�e Gitter mit periodischen Randbedingungen bei 10000 Messpunkten pro Temperaturpunkt; {\itshape Quelle:} Eigenwerk}
  \label{fig:KMCSMagnetisierungUndSuszeptibilitaet}
\end{figure}

Die weiter oben angesprochene (dekorrelierte) Gleichverteilung der Spins bei hohen Temperaturen, dr�ckt sich verst�ndlicherweise in der verschwindenen, mittleren {\bfseries Magnetisierung} (Gl. \ref{eq:Magnetisierung}) in Abb. \ref{fig:KMCSMagnetisierung} f�r $t\gg T_c$ aus. Verringert man vom thermischen Chaos aus die Temperatur, so beginnen die Spins in der N�he des kritischen Punktes, sich innerhalb der Wei�schen Bezirke in eine Richtung auszurichten. Da hierbei keine der beiden Richtungen einen Vorzug erh�lt, wechselt der Mittelwert der Magnetisierung beliebig das Vorzeichen.

Hier zeigt sich die Simulation mit dem Metropolis Algorithmus (Gl. \ref{eq:KanonischerMetropolis}) fehlerbehaftet, da sich die positiven und negativen Beitr�ge dieser Bezirke aus Symetriegr�nden insgesamt aufheben m�ssten. Die Cluster kann der {\bfseries Algorithmus} jedoch nur von den R�ndern her umdrehen, denn das Flippen eine Spins in der Mitte eines solchen Clusters ist sehr unwahrscheinlich. Die Cluster bleiben also f�r l�ngere Zeit bestehen und brechen die Symetrie der Verteilung -- es entsteht eine {\bfseries vermeintliche Magnetisierung}, die letztlich dem Verlust von Ergodizit�t (siehe Abschnitt \ref{sec:Ergodizitaet}) der Markov-Kette geschuldet ist. F�r kleine Temperaturen sieht man sogar, dass sich das Vorzeichen garnicht mehr ver�ndert und mit der Zeit alle Spins ebenfalls passend geflippt werden.

Ein Beispiel f�r eine Modifikation des Algorithmus', welcher diesem Sachverhalt Sorge tr�gt, ist der Cluster-Algorithmus, der von Swendson/Wang (1987) und Wolff (1989)  erstmals vorgestell wurde. Die Cluster werden analysiert, im Ganzen gewichtet und eventuell geflippt. F�r kleine Temperaturen werden die Cluster also korrekterweise ebenfalls bei fast jedem MC-Schritt geflippt und die Mittelung ergibt wie erwartet eine verschwindende Magnetisierung (vgl. \cite{Diplom}).

Die mittlere magnetische Suszeptibilit�t (Gl. \ref{eq:Suszeptibilitaet}) in Abb. \ref{fig:KMCSSuszeptibilitaet} zeigt ein �hnliches Verhalten, wie die W�rmekapazit�t. Auch sie formt zum Phasen�bergang hin einen einen Peak aus. In der Realit�t ist allerdings auch sie immer 0, falls es kein �u�eres Magnetfeld gibt.

\subsection{Mittelwert der abs. Magnetisierung und mag. Suszeptibilit�t}

Nachdem die Magnetisierung zuf�llig das Vorzeichen wechselt und sich f�r Systeme ohne externes Magnetfeld wegmittelt, betrachtet man h�ufig nur die {\bfseries absolute Magnetisierung} (Gl. \ref{eq:AbsoluteMagnetisierung}) in Abb. \ref{fig:KMCSAbsoluteMagnetisierung} und deren Varianz (Gl. \ref{eq:AbsoluteSuszeptibilitaet}) in Abb. \ref{fig:KMCSAbsoluteSuszeptibilitaet}. Beide Gr��en stellen sogenannte {\bfseries Ordnungsparameter} dar; sie verdeutlichen gut das Ausbilden von Clustern. Insbesondere zeigen sie f�r {\bfseries verschiedene Systemgr��en} $N$, dass der Prozess vom thermischen Chaos zur Ordnung hin im gr��eren System deutlich schneller vonstatten geht. Da sie Spin-Inversions-invariant sind, stellen sie trotz Clusterbildung Gr��en dar, welche der Metropolis Algorithmus zuverl�ssig berechnet.

Die mittlere absolute magnetische Suszeptibilit�t bietet sich neben der W�rmekapazit�t weiterhin an, $T_c$ und kritische Exponenten zu finden \ref{sec:IsingEnergie}.

\begin{figure}[bh]
  \centering
  \subfloat[{\bfseries Mittelwerte der abs. Magnetisierung}]{
    \label{fig:KMCSAbsoluteMagnetisierung}
    \includegraphics[width=0.48\textwidth]{Diagramme/KMCS/AbsoluteMagnetisierung-Temperatur} 
  }
  \subfloat[{\bfseries Mittelwerte der abs. mag. Suszeptibilit�t}]{
    \label{fig:KMCSAbsoluteSuszeptibilitaet}
    \includegraphics[width=0.48\textwidth]{Diagramme/KMCS/AbsoluteMagnetischeSuszeptibilitaet-Temperatur} 
  }
  \caption[Mittlere abs. Magnetisierung und mag. Suszeptibilit�t; {\itshape Quelle:} Eigenwerk]{Mittelwerte der absoluten Magnetisierung und magnetischen Suszeptibilit�t f�r verschieden gro�e Gitter mit periodischen Randbedingungen bei 10000 Messpunkten pro Temperaturpunkt; {\itshape Quelle:} Eigenwerk}
  \label{fig:KMCSAbsoluteMagnetisierungUndSuszeptibilitaet}
\end{figure}

\chapter[Quantenmechanische MCS mit Hilfe der Stochastic Series Expansion]{Quantenmechanische MCS\\\LARGE mit Hilfe der Stochastic Series Expansion}

Im Folgenden wollen wir uns nun der Simulation des quantenmechanischen Spin-1/2 Heisenberg Systems zuwenden. Die verwendete Methode {\itshape Stochastic Series Expansion} hat ihren Ursprung in der 1962 vorgestellten Reihenentwicklung von D.C Handscomb \cite{Handscomb}, welche seit 1991 besonders von Anders W. Sandvik, Olaf F. Sylju\aa sen und Juhani Kurkij�rvi weiterentwickelt wurde \cite{Diplom}.

Unsere Beispielapplikation wird f�r verschiedene Systeme wie {\itshape Offene Kette} oder {\itshape Periodisches Gitter} die Energie und die W�rmekapazit�t berechnen, wobei wir s�mtliche Daten immer im Vergleich zur Exakten Diagonalisierung des Hamiltonoperators betrachten und einordnen wollen.

\section{Methode}

\subsection{Das Spin-1/2 Heisenberg System}

Der Hamiltonian des Spin-1/2 Heisenberg Systems mit Zeemann-Term ist gegeben durch

\begin{equation}
H_{\mathrm{Heisenberg}}=\sum_{\left\langle i,j\right\rangle}J_{ij}\cdot\left(S_i^xS_j^x+S_i^yS_j^y+\Delta S_i^zS_j^z\right)-h\sum_{i=0}^{N-1}\mu_{i}\cdot S_i^z\ \mathrm{.}
\label{eq:HeisenbergHamiltonian}
\end{equation}

Man betrachtet also eine 3-dimensionale Koppelung von benachbarten Spins (gewichtet in $Z$-Richtung) mit der Bindungsmatrix $\boldsymbol{J}$ an der zus�tzlich ein externes Magnetfeld $\boldsymbol{h}=(0,0,h)^T$ angreift. Das Magnetische Moment ist mit $\boldsymbol{\mu}=(0,0,\mu)^T$ benannt.

Das Heisenberg System wird folgenderma�en nach $\Delta$ klassifiziert:

\begin{itemize}
\item $\Delta<-1$: Ising-Phase
\item $\Delta=-1$: Isotrope ferromagnetische Phase
\item $\vert\Delta\vert=1$: XY-Phase
\item $\Delta=1$: Isotrope antiferromagnetische Phase
\item $\Delta>1$: N\'eel-Phase
\end{itemize}

Wir wollen uns hier speziell mit der isotropen antiferromagnetischen Phase besch�ftigen, wobei wir auch hier wieder jedes $J_{ij}=1$ sowie $\mu_i=1$ setzen wollen. Die Anordnung betrachten wir weiterhin ohne Magnetfeld ($h=0$). Daraus ergiebt sich der vereinfachte Hamiltonian

\begin{equation}
H=\sum_{\left\langle i,j\right\rangle}S_i^xS_j^x+S_i^yS_j^y+S_i^zS_j^z=\sum_{\left\langle i,j\right\rangle}\boldsymbol{S}_i\cdot\boldsymbol{S}_j
\label{eq:BeispielHeisenbergHamiltonian}
\end{equation}

welchen man mit $S_i^\pm=S_i^x\pm iS_i^y$ zu

\begin{equation}
H=\sum_{b=1}^{N_b}\frac{1}{2}\left(S_{i(b)}^+S_{j(b)}^-+S_{i(b)}^-S_{j(b)}^+\right)+S_{i(b)}^zS_{j(b)}^z
\label{eq:BeispielHeisenbergHamiltonianPM}
\end{equation}

umformen kann, wobei wir die P�rchen $\left\langle i,j\right\rangle$ durchnummeriert haben und f�r die Spin-Indices Nachbar-Funktionen $i(b)$ und $j(b)$ verwenden. Diese ergeben sich aus der Systembeschaffenheit. Betrachtet man die Spin Operatoren dann in der Standardbasis bez�glich $S^z$, so kann man den Hamiltonian pro Koppelung (engl. Bond) in einen diagonalen $H_{0,b}$ und einen off-diagonalen Teil $H_{1,b}$ aufspalten,

\begin{equation}
H=-\sum_{b=1}^{N_b}\left(\underbrace{\frac{1}{4}-S_{i(b)}^zS_{j(b)}^z}_{H_{0,b}}-\underbrace{\frac{1}{2}\left(S_{i(b)}^+S_{j(b)}^-+S_{i(b)}^-S_{j(b)}^+\right)}_{H_{1,b}}\right) + \left\{\frac{N_b}{4}\right\}\ \mathrm{.}
\label{eq:BeispielHeisenbergHamiltonianTeile}
\end{equation}

Aus einem bald ersichtlichen Grund f�hren wir dar�ber hinaus eine Konstante $1/4$ ein, welche wir der Energie am Schluss wieder hinzuf�gen werden.

\subsection{Reihenentwicklung}

Wie beim klassischen Ising Modell im Abschnitt \ref{sec:KanonischeUebergangswahrscheinlichkeiten} versuchen wir nun auch, die Zustandssumme einfacher zu erhalten, als den gesamten Zustandsraum abtasten zu m�ssen. Der obere Ansatz ist hier jedoch nicht hilfreich, weil wir die Energie bzw. den Hamiltonian f�r den Boltzmannfaktor (in der Zustandssumme) nicht berechnen k�nnen. Deshalb schlagen wir einen anderen Weg ein:

Die quantenmechanische Zustandssumme

\begin{equation}
Z=\tr e^{-\beta H}
\label{eq:QuantenmechanischeZustandssumme}
\end{equation}

ist �ber Spur der "`Boltzmannmartix"' definiert. Schreiben wir die Spur mit der Basis $\vert\alpha\rangle$ und verwenden f�r die Exponentialfunktion die Reihendarstellung, ergibt sich

\begin{align}
Z&=\sum_{n=0}^\infty\frac{(-\beta)^n}{n!}\sum_\alpha\langle\alpha\mid H^n\mid\alpha\rangle\label{eq:QuantenmechanischeZustandssummeReihe}\\
&=\sum_{n=0}^\infty\frac{(-\beta)^n}{n!}\sum_\alpha\langle\alpha\mid\left(-\sum_{b=1}^{N_b}H_{0,b}-H_{1,b}\right)^n\mid\alpha\rangle\label{eq:QuantenmechanischeZustandssummeHEingesetzt}\ \mathrm{,}
\end{align}

wobei wir die obige Gl. \ref{eq:BeispielHeisenbergHamiltonianTeile} f�r den Hamiltonian einsetzen.

An dieser Stelle f�hren wir die Potenzierung der Hamiltonians explizit aus und erhalten dadurch $n$-lange Hamiltonoperatorketten, sogenannte Operatorstrings. Jedes Kettenglied ist hierbei entweder diagonal oder off-diagonal und geh�rt zu einer bestimmten Koppelung $b$. Da jede m�gliche Kombination von $n$ Operatoren vorkommt, haben wir insgesamt $(2N_b)^n$ Operatorstrings. Um diese geeignet zu verwalten, definieren wir die Menge aller $n$-langen Operatorstrings $\{S_n\}$ und ordnen jedem String je eine Funktion f�r die beiden Indizes $a(p)\in\{0,1\}$ und $b(p)\in\{1,\ldots,N_b\}$ der Hamiltonoperatoren zu, wobei $p$ die Position im String darstellt. Dar�ber hinaus hebt sich das erste Minuszeichen vor der Summe mit dem vor dem $\beta$ weg, das Minuszeichen zwischen den Hamiltonoperatoren ziehen wir mit der Anzahl der off-diagonalen Operatoren $n_1$ vor das Matrixelement:

\begin{equation}
Z=\sum_{n=0}^\infty\frac{\beta^n}{n!}\sum_\alpha\sum_{\{S_n\}}(-1)^{n_1}\langle\alpha\mid\prod_{p=0}^{n-1}H_{a(p),b(p)}\mid\alpha\rangle
\label{eq:QuantenmechanischeZustandssummeStrings}
\end{equation}

$S_n$ bestimmt also durch $a(p)$ und $b(p)$ die Zuordnung der Indizes auf den $p$-ten Operator.

Nun betrachten wir noch die unendliche Summe �ber $n$. Diese schneiden wir bei $n=L$ ab (eine Fehlerrechnung folgt sp�ter) und bringen alle Operatorstrings mit $n<L$ auf die L�nge $L$, indem wir $L-n$ Einheitsmatrizen in sie einf�gen, die wir sinnvollerweise $H_{0,0}$ nennen. Dies f�hrt zu nur noch {\bfseries einer} Menge von Operatorstrings $S_L$, in die die k�rzeren integriert wurden. Da es aber $\binom{L}{n}$ M�glichkeiten gibt die Einheitsmatrizen einzuf�gen, m�ssen wir hierdurch teilen, da ein ehemaliger $S_n$ Operatorstring auch zuk�nftig nur einfach in die Zustandssumme eingehen soll:

\begin{equation}
Z=\sum_{\{S_L\}}\frac{\beta^{n}(-1)^{n_1}(L-n)!}{L!}\sum_\alpha\langle\alpha\mid\prod_{p=0}^{L-1}H_{a(p),b(p)}\mid\alpha\rangle
\label{eq:QuantenmechanischeZustandssummeAbschneiden}
\end{equation}

$n$ ist nun nicht mehr die Stringl�nge, sondern die Anzahl der Operatoren ungleich $H_{0,0}$.

Wir wollen uns nun die Wirkung des Operatorstrings auf die Basis $\vert\alpha\rangle$ ansehen: Diese {\itshape propagierten Zust�nde}

\begin{equation}
\vert\alpha(Q)\rangle=\prod_{p=0}^{Q-1}H_{a(p),b(p)}\mid\alpha\rangle
\label{eq:PropagierteZust�nde}
\end{equation}

sind neue Basiszust�nde, sie ergeben sich also nicht aus der Superposition von anderen Zust�nden. Explizit ist die Wirkung eines diagonalen Operators auf einen Basiszustand

\begin{align}
H_{0,b}\mid\ldots\uparrow_{i(b)}\ldots\uparrow_{j(b)}\ldots\rangle&=\frac{1}{4}-(\frac{1}{2}\cdot\frac{1}{2})=0\ \mathrm{,}\label{eq:DigonalOpAuf11}\\
H_{0,b}\mid\ldots\downarrow_{i(b)}\ldots\downarrow_{j(b)}\ldots\rangle&=\frac{1}{4}-(-\frac{1}{2}\cdot-\frac{1}{2})=0\ \mathrm{,}\label{eq:DigonalOpAuf00}\\
\langle\ldots\uparrow_{i(b)}\ldots\downarrow_{j(b)}\ldots\mid H_{0,b}\mid\ldots\uparrow_{i(b)}\ldots\downarrow_{j(b)}\ldots\rangle&=\frac{1}{4}-(\frac{1}{2}\cdot-\frac{1}{2})=\frac{1}{2}\ \mathrm{,}\label{eq:DigonalOpAuf10}\\
\langle\ldots\downarrow_{i(b)}\ldots\uparrow_{j(b)}\ldots\mid H_{0,b}\mid\ldots\downarrow_{i(b)}\ldots\uparrow_{j(b)}\ldots\rangle&=\frac{1}{4}-(-\frac{1}{2}\cdot\frac{1}{2})=\frac{1}{2}\label{eq:DigonalOpAuf01}
\end{align}

und die Wirkung eines off-diagonalen Operators auf einen Basiszustand

\begin{align}
H_{1,b}\mid\ldots\uparrow_{i(b)}\ldots\uparrow_{j(b)}\ldots\rangle&=\frac{1}{2}(0+0)=0\ \mathrm{,}\label{eq:OffDigonalOpAuf11}\\
H_{1,b}\mid\ldots\downarrow_{i(b)}\ldots\downarrow_{j(b)}\ldots\rangle&=\frac{1}{2}(0+0)=0\ \mathrm{,}\label{eq:OffDigonalOpAuf00}\\
\langle\ldots\downarrow_{i(b)}\ldots\uparrow_{j(b)}\ldots\mid H_{0,b}\mid\ldots\uparrow_{i(b)}\ldots\downarrow_{j(b)}\ldots\rangle&=\frac{1}{2}\ \mathrm{,}\label{eq:OffDigonalOpAuf10}\\
\langle\ldots\uparrow_{i(b)}\ldots\downarrow_{j(b)}\ldots\mid H_{0,b}\mid\ldots\downarrow_{i(b)}\ldots\uparrow_{j(b)}\ldots\rangle&=\frac{1}{2}\ \mathrm{.}\label{eq:OffDigonalOpAuf01}
\end{align}

Dieser Sachverhalt vereinfacht den Algorithmus deutlich. Beim Sampling sp�ter tragen also nur Operatorstrings bei, deren Operatoren auf nicht-parallele Spins wirken. Das Matrixelement einer solchen Operation ist au�erdem immer $\frac{1}{2}$. Dies war der Grund f�r die Konstante $\frac{1}{4}$ in $H_{0,b}$ aus Gl. \ref{eq:BeispielHeisenbergHamiltonianTeile}.

Das Gewicht f�r eine beitragende Konfiguration $\sigma$ ist also

\begin{equation}
W(\sigma,S_L)=\left(\frac{\beta}{2}\right)^n\frac{(L-n)!}{L!}
\label{eq:HeisenbergGewichte}
\end{equation}

wobei wir verwenden, dass $n_1$ auf Quadratgittern und Ketten immer gerade ist und deshalb wegf�llt.

\subsection{Sampling}

Zu Beginn der Simulation gehen wir von einem leeren Operatorstring und einer zuf�lligen Spinanordnung aus und f�hren wie in Kapitel \ref{sec:Ising} MC-Schritte aus. Diese sind unterteilt in ein Diagonal Update, welches diagonale Operatoren in den String ein- und ausbaut und in ein off-diagonales Loop Update, welches diagonale zu off-diagonale Operatoren hin- und zur�cktransformiert. Jedes Update muss allerdings darauf achten, dass die oben angegeben Beschr�nkungen nicht verletzt werden:

\begin{itemize}
\item Operatoren d�rfen nicht auf parallele Spin-Paare wirken, ansonsten zerst�ren sie den Zustand (Lokale Bedingung).
\item Die Periodizit�t $\vert\alpha\rangle=\vert\alpha(0)\rangle=\vert\alpha(L)\rangle$ des Algorithmus' muss gewahrt werden.
\end{itemize}

Das Sampling kann durch Graphiken wie \ref{fig:OffDiagonalNormal} veranschaulicht werden. Die Anfangskonfiguration $\vert\alpha(0)\rangle$ steht hierbei unten und propagiert durch den Operatorstring in den Endzustand $\vert\alpha(L)\rangle=\vert\alpha(0)\rangle$. Operatoren sitzen jeweils auf zwei "`Spin-Bahnen"' und lassen diese entweder unber�hrt (diagonale Operatoren, wei� dargestellt) oder vertauschen deren Ausrichtung (off-diagonale Operatoren, schwarz dargestellt). Reihen ohne Operatoren signalisieren einen $H_{0,0}$ Operator, also eine Einheitsmatrix.

\begin{figure}[thb]
  \centering
  \includegraphics[width=0.3\textwidth]{Bilder/Off-Diagonal-Normal} 
  \caption[Visualisierung des Operatorstrings]{{\bfseries Visualisierung des Operatorstrings:} die Kreise ganz unten und ganz oben stellen die Spinkonfiguration $\vert\alpha(0)\rangle$ bzw. $\vert\alpha(L)\rangle$ dar. Dazwischen liegen diagonale Operatoren (wei�) und off-diagonale (schwarz). Reihen ohne Operatoren signalisieren einen $H_{0,0}$ Operator. Quelle: \cite{Sandvik}}
  \label{fig:OffDiagonalNormal}
\end{figure}

\subsubsection{Diagonales Update}

F�r das Einf�gen von diagonalen Operatoren in den String, muss darauf geachtet werden, dass dieser nicht an parallele Spins angelegt wird. Die Periodizit�t wird durch die Aktion nicht gest�rt, da diagonale Operatoren den Zustand nicht ver�ndern. F�r das Entfernen von diagonalen Operatoren aus dem Operatorstring ist nicht einmal die erste Bedingung problematisch.

Die Wahrscheinlichkeit f�r das Einf�gen eines Operators an einem Platz $p$ gibt man analog zur Gl. \ref{eq:Metropolis} mit

\begin{equation}
W_{\nu\sigma,\ \mathrm{Einf"ugen}}=\begin{cases}
\frac{W(\sigma,S_L)\cdot N_b}{W(\nu,S_L)}=\frac{\beta N_b}{2(L-n)} & W(\sigma,S_L)\cdot N_b<W(\nu,S_L)\\
1                                                                  & W(\sigma,S_L)\cdot N_b\geq W(\nu,S_L)
\end{cases}
\label{eq:EinfuegeWahrscheinlichkeit}
\end{equation}

an. Analog f�r dies erh�lt man beim L�schens eines Operators (Ersetzen mit der Einheitsmatrix)

\begin{equation}
W_{\nu\sigma,\ \mathrm{Entfernen}}=\begin{cases}
\frac{W(\sigma,S_L)\cdot N_b}{W(\nu,S_L)}=\frac{2(L-n+1)}{\beta N_b} & W(\sigma,S_L)\cdot N_b<W(\nu,S_L)\\
1                                                                    & W(\sigma,S_L)\cdot N_b\geq W(\nu,S_L)\ \mathrm{.}
\end{cases}
\label{eq:EntferneWahrscheinlichkeit}
\end{equation}

Die Eigenschaft {\itshape Detailed Balance} kann hier leicht durch Einsetzen in \ref{eq:DetailedBalance} gepr�ft werden.

\subsubsection{Off-Diagonales Update}

Beim Austauschen von diagonalen und off-diagonalen Operatoren ist der Sachverhalt etwas komplizierter, denn dadurch wird eine Vertauschung von Spinausrichtungen (durch den off-diagonalen Operator) hinzugef�gt bzw. entfernt. Es ist selbstverst�ndlich, dass unter der Ber�cksichtigung der Periodizit�t also mindestens zwei off-diagonale Operatoren in solch eine Aktion involviert sein m�ssen. Befinden sich zwischen diesem Operator-Paar allerdings weitere Operatoren, kann es zu einer Regelverletztung der lokalen Bedingung kommen, da ein �ndern der Zust�nde zwischen dem Operatorpaar den dazwischen beeinflusst (s. Abb. \ref{fig:Off-Diagonal-Works}). In Abb. \ref{fig:Off-Diagonal-NeedsLoop} wird eine m�gliche L�sung dieses Problems dargestellt, hier wird nicht der Zustand zwischen dem Operatorenpaar ver�ndert, sondern der au�erhalb (inklusive dem Anfangs- und Endzustand).

\begin{figure}[thb]
  \centering
  \subfloat[{\bfseries Problematischer (gestrichelt) und Unproblematischer Fall (durchgezogen)}]{
    \label{fig:Off-Diagonal-Works}
    \includegraphics[width=0.48\textwidth]{Bilder/Off-Diagonal-Works} 
  }
  \quad
  \quad
  \subfloat[{\bfseries Ausweg �ber die Modifikation des Anfangszustand}]{
    \label{fig:Off-Diagonal-NeedsLoop}
    \includegraphics[width=0.48\textwidth]{Bilder/Off-Diagonal-NeedsLoop} 
  }
  \caption[Darstellung m�g. Probleme bzgl. verletzter Bedingungen beim Off-Diag. Update]{Darstellung m�glicher Probleme bzgl. verletzter Bedingungen beim Off-Diagonalen Update. In (a) sehen wir wie das durchgezogen marktierte Update unproblematisch durchgef�hrt, wobei das gestrichelte Update den Operator zwischen dem Paar so beeinflussen w�rde, dass dieser die lokale Bedingung nicht mehr erf�llt. In (b) sehen wir einen m�glichen Ausweg, da hier Zust�nde zwischen den Operatoren nicht ge�ndert werden, sondern der Anfangs- bzw. Endzustand. Quelle: \cite{Sandvik}}
  \label{fig:ProblematikVonOffDiagonalUpdates}
\end{figure}

Eine andere M�glichkeit w�re nat�rlich, den zweiten (links) anliegenden Spin des involvierten mittleren Operators auch zu �ndern, sodass die lokale Bedingung wieder erf�llt ist. Dies beeinflusst aber wieder andere Operatoren, etc.. Im Endeffekt versucht man also alle sogennanten Loops (s. Abb. \ref{fig:Off-Diagonal-Loop}), abgeschlossene Wege durch den Operatorstring, zu finden, da man diese dann wie in Abb. \ref{fig:Off-Diagonal-LoopDone} unabh�ngig von einander flippen kann (Loop Update), wobei Operatoren die ganz in der Loop liegen hin- und sogleich wieder zur�ckgeflippt werden. Der Ausweg von Abb. \ref{fig:Off-Diagonal-NeedsLoop} ist in dieser L�sung inbegriffen, da Loops auch �ber den periodischen Rand hinaus f�hren k�nnen.

\begin{figure}[thb]
  \centering
  \subfloat[{\bfseries Operatorstring mit eingezeichneter Loop}]{
    \label{fig:Off-Diagonal-Loop}
    \includegraphics[width=0.3\textwidth]{Bilder/Off-Diagonal-Loop} 
  }
  \quad
  \quad
  \subfloat[{\bfseries Operatorstring nach dem Flip der Loop}]{
    \label{fig:Off-Diagonal-LoopDone}
    \includegraphics[width=0.3\textwidth]{Bilder/Off-Diagonal-LoopDone} 
  }
  \caption[Darstellung einer Loop im Operatorstring]{Darstellung einer Loop im Operatorstring. (a) bezieht sich noch auf den Ausgangszustand, (b) ergiebt sich nach dem flippen der in (a) angegeben Loop. Operatoren derer beider Seiten am selben Loop liegen werden nicht geflippt (hin und wieder zur�ckgeflippt). Quelle: \cite{Sandvik}}
  \label{fig:LoesungDurchLoopUpdate}
\end{figure}

\subsection{Formeln f�r die mittlere Energie und W�rmekapazit�t}

\subsubsection{Energie}

Augehend von der Gl. \ref{eq:QuantenmechanischeZustandssummeReihe} wollen wir nun eine Formel f�r die mittlere Energie pro Spin $E/N$ herleiten,

\begin{equation}
Z=\sum_{n=0}^\infty\frac{(-\beta)^n}{n!}\sum_{\{\alpha\}_n}\langle\alpha_0\mid H\mid\alpha_{n-1}\cdots\langle\alpha_1\mid H\mid\alpha_0\rangle\ \mathrm{,}
\label{eq:QuantenmechanischeZustandssummeReiheAusgeschrieben}
\end{equation}

wobei in die hintere Summe $n-1$ Summen �ber die Basis eingef�gt wurden. Sodann ergiebt sich der Mittelwert von $E/N$ mit

\begin{align}
\frac{E}{N}&=\frac{1}{ZN}\sum_{n=0}^\infty\frac{(-\beta)^n}{n!}\sum_{\{\alpha\}_{n+1}}\langle\alpha_0\mid H\mid\alpha_n\cdots\langle\alpha_1\mid H\mid\alpha_0\rangle\label{eq:QuantenEnergie}\\
           &=\frac{1}{ZN}\sum_{n=1}^\infty\frac{(-\beta)^n}{n!}\frac{n}{-\beta}\sum_{\{\alpha\}_n}\langle\alpha_0\mid H\mid\alpha_{n-1}\cdots\langle\alpha_1\mid H\mid\alpha_0\rangle\label{eq:QuantenEnergieSubstitution}\\
           &=\frac{1}{ZN}\sum_{n=0}^\infty\frac{(-\beta)^n}{n!}\frac{n}{-\beta}\sum_{\{\alpha\}_n}\langle\alpha_0\mid H\mid\alpha_{n-1}\cdots\langle\alpha_1\mid H\mid\alpha_0\rangle\label{eq:QuantenEnergieMit0}\\
           &=-\frac{\langle n\rangle}{N\beta}\label{eq:QuantenEnergieMittelwert}\\
\left(\frac{E}{N}\right)_{\mathrm{Real}}&=-\frac{\langle n\rangle}{N\beta}+\left\{\frac{N_b}{4N}\right\}\ \mathrm{.}\label{eq:QuantenEnergieMitShift}
\end{align}

F�r die erste Form sollte man bemerken, dass die letzte Summe nun �ber ein $H$ mehr l�uft. Die erste Umformung substituiert $n:=n+1$, die zweite f�gt den $n=0$-Term wieder ein, dies ist m�glich, da er sowieso 0 ergibt. Sodann kann erkannt werden, dass es sich bei dem vorliegenden Ausdruck um den Mittelwert von $n$ handelt. Zu guter Letzt f�gen wir noch die "`verlorene"' Energie-Konstante von Gl. \ref{eq:BeispielHeisenbergHamiltonianTeile} hinzu.

\subsubsection{W�rmekapazit�t}

Die W�rmekapazit�t pro Spin erh�lt man dann �ber die Ableitung der mittleren Energie nach der Temperatur,

\begin{align}
\frac{C}{N}&=\frac{\partial_T E}{N}\\
           &=-\frac{1}{NT}\partial_T \langle n\rangle-\frac{\langle n\rangle}{N}\label{eq:QuantenWaermeKapazitaet}\\
           &=\frac{\langle n^2\rangle-\langle n\rangle^2-\langle n\rangle}{N}\label{eq:QuantenWaermeKapazitaetBest}\ \mathrm{.}
\end{align}

\subsection{Cut-Off L}
\label{sec:cutoff}

Da f�r $T\rightarrow\infty$ $C\rightarrow0$ sehen wir, dass $\var n=\langle n\rangle$. D.h. dass der $T$ -- $n$ Graph in beide Richtungen exponentiell abf�llt. Nach \cite{Sandvik} kann f�r $L$ also ein genug h�herer Wert als $\langle n\rangle$ verwendet (ca. $4/3\cdot\langle n\rangle$). $n$ erreicht $L$ dann praktisch nie.

Im Algorithmus muss $L$ also f�r jeden MC-Schritt �berpr�ft werden und gegebenfalls angepasst werden.

\section{Implementierung}

Die Struktur der Anwendung ist auch wieder angelehnt an die Beschreibung in \cite{Sandvik}. Der Algorithmus �hnelt dem der Klassischen MCS in Abschnitt \ref{sec:KlassischeImplementierung} sehr, verwendet allerdings andere Messformeln (s. Gl. \ref{eq:QuantenEnergieMitShift} und \ref{eq:QuantenWaermeKapazitaetBest}) und einen anderen MC-Schritt, da er den Operatorstring inklusive Anfangskonfiguration samplet und nicht nur die Einzelnen Konfigurationen.

\subsection{Initialisierung}

Wie bei der Klassischen MCS be�tigen wir zuerst die Eingabeparameter:

\begin{itemize}
\item Anzahl der Spins $N$,
\item Anzahl der Messungen $\widetilde{R}$ und
\item Temperatur des Systems $T$.
\end{itemize}

Anschlie�end legen wir ein $N$-langen Bitfeld, welches unseren Anfangszustand $\vert\alpha(0)\rangle$ enth�lt, und initialisieren es mit zuf�lligen Daten. Um den aktuellen Operatorstring abspeichern zu k�nnen legen wir weiterhin ein Integer-Array $s$ $L$-ter L�nge an, welches f�r jeden Operatorplatz $p$ die Art des Operators (diagonal, off-diagonal oder Einheitsmatrix) -- also $a(p)$ und dessen Position im System $b(p)$ (Koppelung) lagert. Dies k�nnen wir gemeinsam in einer Ganzzahl speichern, wenn wir ausnutzen, dass $a\in\{0,1\}$:

\begin{equation}
s(p)=a(p)+2b(p)
\label{eq:OperatorArrayStorage}
\end{equation}

wir k�nnen die Informationen aus der Ganzzahl $s(p)$ wieder erhalten, wenn wir pr�fen, ob

\begin{itemize}
\item $s(p)=0 \Rightarrow$ Einheitsmatrix,
\item $\even s(p) \Rightarrow$ Diagonaler Operator,
\item $\odd s(p) \Rightarrow$ Off-diagonaler Operator.
\end{itemize}

Die Position $b(p)$ des Operators erhalten wir, wenn wir mittels einer ganzzahligen Division $b(p)=s(p)/2$. Au�erdem speichern wir die Anzahl der Operatoren $n$, die nicht eine Einheitsmatrix darstellen, da diese Gr��e sp�ter zum Berechnen unserer Messgr��en verwendet wird.

\subsection{Simulation}

F�r jeden MC-Schritt wird zuerst ein Diagonal Update durchgef�hrt und anschlie�end ein Off-Diagonal Loop Update. Am Schluss wird der Cut-Off $L$ nach Abschnitt \ref{sec:cutoff} eventuell weiter noch oben gesetzt, um dem System genug Freiraum zum Einf�gen weiterer Operatoren zu geben. D.h. wir verl�ngern den Operatorstring, um immer genug Einheitsmatrizen frei f�r diagonale Operatoren zu haben.

\subsubsection{Diagonal Update}

Bei jedem Diagonal Update wird f�r jeden Operatorplatz $p$, auf dem bereits ein diagonaler Operator sitzt, mittels einem Zufallstest entschieden, ob der Operator durch eine Einheitsmatrix ersetzt werden darf. Ist eine Zufallszahl kleiner als die Wahrscheinlichkeit aus Gl. \ref{eq:EntferneWahrscheinlichkeit}, verringert man $n$ um 1 und vermerkt im Operatorstring $s(p)=0$.

Existiert noch kein Operator auf der Position, versucht man einen diagonalen Operator einzuf�gen: Die Koppelung $b$ des Operators wird zuf�llig bestimmt. Wenn die beiden Spins an dieser Koppelung anti-parallel sind, wird mit einem �hnlichen Zufallstest wie oben mit der Wahrscheinlichkeit aus Gl. \ref{eq:EinfuegeWahrscheinlichkeit} entschieden, ob das Einf�gen erfolgt. Als Konsquenz w�rden wir $n$ um 1 erh�hen und $s(p)=2b$ setzen. Um den bis zu $p$ propagierten Zustand $\vert\alpha(p)\rangle$ schnell zu erhalten, schreiben wir ihn in jedem $p$-Schritt mit.

Trifft man auf einen Off-Diagonalen Operator wird nichts am Operatorstring ver�ndert, nur der propagierte Zustand wird angepasst (die beiden Spins vertauschen ihren Zustand).

\subsubsection{Off-Diagonal Loop Update}

Um nun die durch das Diagonal Update eingef�gten Operatoren auch in Off-Diagonale Operatoren zu transformieren (oder zur�ck), wird der Operatorstring nun systematisch nach Loops gescannt und f�r jeden Loop wird anschlie�end mit der Wahrscheinlichkeit 50\% entschieden, diese gesamte Loop zu flippen.

\paragraph{Die Analyse des Operatorstrings} kann wie folgt geschehen:

Jeder Operator (der keine Einheitsmatrix ist) besitzt vier Vertizes, das sind die Ankn�pfungspunkte an die Spin-Bahnen (zwei oben, zwei unten). Diese erhalten nach Abb. \ref{fig:Vertizes} je eine Typnummer $y\in\{0,1,2,3\}$. Mithilfe dieses $y$ und der Position des Operators im String $p$ kann man jeden Vertex global mit der Zahl $v$ indizieren:

\begin{equation}
v(y,p)=v+4p
\label{eq:OperatorVertexStorage}
\end{equation}

Von der Ganzzahl $v$ kann man wie bei \ref{eq:OperatorArrayStorage} wieder auf die speziellen eigenschaften schlie�en.

\begin{figure}[thb]
  \centering
  \includegraphics[width=0.6\textwidth]{Bilder/Vertizes}
  \caption[M�gliche Verwendung der Operatoren mit den Typnummern der Vertizes]{{\bfseries M�gliche Verwendung der Operatoren mit den Typnummern der Vertizes} Quelle: \cite{Sandvik}}
  \label{fig:Vertizes}
\end{figure}

Nun gehen wir jede Position $p$ im String durch und verkn�pfen Vertizes, die sich auf einer Spin-Bahn gegen�berliegen. Die Verkn�pfungen stellen also z.B. die dicken Linien in der Abb. \ref{fig:Off-Diagonal-Loop} dar, die die Operatoren verbinden. D.h. Vertizes stellen die Ecken eines Loop-Gebiets dar, diese Verkn�pfungen und die Operatoren die Kanten. Die Verkn�pfung wird in einem Verkn�pfungs-Array $x$ gespeichert. Zu beachten ist, dass Loops durchaus auch �ber den periodischen Rand hinweg m�glich sind!

\paragraph{Das Flippen der Loops} wird nun St�ck f�r St�ck durchgef�hrt. F�r jede Loop wird mit einer 50\% Wahrscheinlichkeit entschieden, ob diese geflippt werden soll. Ist dies der Fall, geht man auf den Kanten des Loop-Gebiets von Operator zu Operator und flippt jeden. Operatoren die mitten in einer Loop liegen werden also nicht ver�ndert. Wird eine Loop geflippt, die sich �ber die periodischen R�nder hinaus erstreckt, muss anschlie�end der Anfangszustand demensprechend ver�ndert werden.

\subsection{Analyse}

Die abgespeicherten Werte f�r $n$ werden nach der Simulation verwendet, um sie -- wie in Abschnitt \ref{sec:Autokorrelation} erkl�rt -- zu analysieren. Verwendung zur Berechnung der Gr��en {\itshape Energie pro Spin} und {\itshape W�rmekapazit�t pro Spin} finden die Gleichungen \ref{eq:QuantenEnergieMitShift} und \ref{eq:QuantenWaermeKapazitaetBest}.

\subsection{Quellcode}

Der vom Author geschriebene C++ Quellcode f�r diese Simulation ist im Anhang \ref{sec:code} zu finden. Folgende Dateien sind hierf�r relevant:

\begin{itemize}
\item\ref{code:SIM}: Hauptprogramm
\item\ref{code:AbstractLattice}: Definition eines abstrakten Gittermodells
\item\ref{code:Open1DLattice}: 1-dimensionales Kettenmodell mit offenen Randbedingungen
\item\ref{code:Periodic1DLattice}: 1-dimensionales Kettenmodell mit periodischen Randbedingungen
\item\ref{code:Periodic2DLattice}: 2-dimensionales Gittermodell mit periodischen Randbedingungen
\item\ref{code:AbstractAlgorithm}: Definition eines abstrakten Algorithmus' zur Simulation
\item\ref{code:SSEAlgorithm}: SSE Algorithmus
\item\ref{code:AbstractAnalyzer}: Definition eines abstrakten Analysemoduls
\item\ref{code:SseEnergyAnalyzer}: Analysemodul f�r die Energie (SSE)
\item\ref{code:SseHeatCapacityAnalyzer}: Analysemodul f�r die W�rmekapazit�t (SSE)
\end{itemize}

\section{Ergebnisse und Diskussion}

\subsection{ED}

4er Kette mit Offenen Randbed

\tiny
\[
H=\left(\begin{blockarray}{cccccccccccccccc}
\begin{block}{\{c\}ccccccccccccccc}
0.75  &&&&&&&&&&&&&&& 0\\
\end{block}
\begin{block}{c\{cccc\}ccccccccccc}
& 0.25  & 0.5   & 0     & 0     \\
& 0.5   & -0.25 & 0.5   & 0     \\
& 0     & 0.5   & -0.25 & 0.5   \\
& 0     & 0     & 0.5   & 0.25  \\
\end{block}
\begin{block}{ccccc\{cccccc\}ccccc}
&&&&& 0.25  & 0.5   & 0     & 0     & 0     & 0     \\
&&&&& 0.5   & -0.75 & 0.5   & 0.5   & 0     & 0     \\
&&&&& 0     & 0.5   & -0.25 & 0     & 0.5   & 0     \\
&&&&& 0     & 0.5   & 0     & -0.25 & 0.5   & 0     \\
&&&&& 0     & 0     & 0.5   & 0.5   & -0.75 & 0.5   \\
&&&&& 0     & 0     & 0     & 0     & 0.5   & 0.25  \\
\end{block}
\begin{block}{ccccccccccc\{cccc\}c}
&&&&&&&&&&& 0.25  & 0.5   & 0     & 0     \\
&&&&&&&&&&& 0.5   & -0.25 & 0.5   & 0     \\
&&&&&&&&&&& 0     & 0.5   & -0.25 & 0.5   \\
&&&&&&&&&&& 0     & 0     & 0.5   & 0.25  \\
\end{block}
\begin{block}{ccccccccccccccc\{c\}}
0&&&&&&&&&&&&&&& 0.75  \\
\end{block}
\end{blockarray}\right)\\
\lambda_i
\]
\normalsize

hat die Eigenwerte \{0.75, 0.75, 0.75, 0.75, 0.75, 0.457107, 0.457107, 0.457107, 0.116025, -0.25, -0.25, -0.25, -0.957107, -0.957107, -0.957107, -1.61603\}

\newpage
\thispagestyle{empty}
\cleardoublepage
\chapter{Zusammenfassung}

Die {\bfseries Stochastic Series Expansion} stellt ein m�chtiges Werkzeug f�r das Sampling von Operatorstrings innerhalb eines quantenmechanischen L�sungsansatzes dar. Sie ist f�r gr��ere 2-dimensionale Systeme der schnellste Weg, um die typischen, thermodynamischen Gr��en zu messen und wird hierf�r an mehreren Instituten erfolgreich eingesetzt. Der Algorithmus ist relativ leicht zu implementieren und arbeitet �u�erst speichersparend.

Im Rahmen dieser Arbeit wurde ein Simulationsprogramm geschrieben, welches nicht nur ein blo�es SSE Modul enth�lt, sondern die M�glichkeit anbietet, die SSE-Daten f�r kleine Systeme mit ED {\bfseries Exakt Diagonalisation} zu �berpr�fen, dar�ber hinaus kann ein Zusatzmodul f�r das numerische L�sen des klassischen Ising-Modells eingesetzt werden. Da das Sofware-Projekt objekt-orientiert ausgelegt ist, k�nnen beliebige Komponenten, wie mit einem Baukastensystem zusammengestellt werden. Es ist dadurch sehr flexibel.

Die durchgef�hrten Messungen best�tigten stets die Theorie und liefern besitzen nur einen sehr geringen Fehler. 

Physikalische Sachverhalte werden an mehreren Beispielen/Bildern erkl�hrt und die Abschnitte "`Methode"' in den beiden Projektskapitel 3 und 4 f�hren ausf�hrliche Beschreibungen der 3 Algorithmen an.

\appendix

\chapter{Quellcode}
\label{sec:code}

\section{Hauptprogramm SIM}

\cppcodefile[label=code:SIM]{SIM.cpp}

\section{Gitter Klassen}

\subsection{Abstrakte Gitter}

\cppcodefile[label=code:AbstractLattice]{Classes/Lattice/AbstractLattice.cpp}

\subsection{1D Gitter mit offenen Randbedingungen}

\cppcodefile[label=code:Open1DLattice]{Classes/Lattice/Open1DLattice.cpp}

\subsection{1D Gitter mit periodischen Randbedingungen}

\cppcodefile[label=code:Periodic1DLattice]{Classes/Lattice/Periodic1DLattice.cpp}

\subsection{2D Gitter mit periodischen Randbedingungen}

\cppcodefile[label=code:Periodic2DLattice]{Classes/Lattice/Periodic2DLattice.cpp}

\section{Algorithmus Klassen}

\subsection{Abstrakter Algorithmus}

\cppcodefile[label=code:AbstractAlgorithm]{Classes/Algorithm/AbstractAlgorithm.cpp}

\subsection{ED}

\cppcodefile[label=code:EDAlgorithm]{Classes/Algorithm/EDAlgorithm.cpp}

\subsection{ISING}

\cppcodefile[label=code:ISINGAlgorithm]{Classes/Algorithm/ISINGAlgorithm.cpp}

\subsection{SSE}

\cppcodefile[label=code:SSEAlgorithm]{Classes/Algorithm/SSEAlgorithm.cpp}

\section{Analysemodule}

\subsection{Abstraktes Analysemodul}

\cppcodefile[label=code:AbstractAnalyzer]{Classes/Analyzer/AbstractAnalyzer.cpp}

\subsection{Analysemodul f�r die Energie (Ising)}

\cppcodefile[label=code:IsingEnergyAnalyzer]{Classes/Analyzer/IsingEnergyAnalyzer.cpp}

\subsection{Analysemodul f�r die W�rme Kapazit�t (Ising)}

\cppcodefile[label=code:IsingHeatCapacityAnalyzer]{Classes/Analyzer/IsingHeatCapacityAnalyzer.cpp}

\subsection{Analysemodul f�r die Magnetisierung (Ising)}

\cppcodefile[label=code:IsingMagnetisationAnalyzer]{Classes/Analyzer/IsingMagnetisationAnalyzer.cpp}

\subsection{Analysemodul f�r die Magnetische Suszeptibilit�t (Ising)}

\cppcodefile[label=code:IsingSusceptibilityAnalyzer]{Classes/Analyzer/IsingSusceptibilityAnalyzer.cpp}

\subsection{Analysemodul f�r die Abs. Magnetisierung (Ising)}

\cppcodefile[label=code:IsingAbsoluteMagnetisationAnalyzer]{Classes/Analyzer/IsingAbsoluteMagnetisationAnalyzer.cpp}

\subsection{Analysemodul f�r die Abs., Mag. Suszeptibilit�t (Ising)}

\cppcodefile[label=code:IsingAbsoluteSusceptibilityAnalyzer]{Classes/Analyzer/IsingAbsoluteSusceptibilityAnalyzer.cpp}

\subsection{Analysemodul f�r die Energie (SSE)}

\cppcodefile[label=code:SseEnergyAnalyzer]{Classes/Analyzer/SseEnergyAnalyzer.cpp}

\subsection{Analysemodul f�r die W�rme Kapazit�t (SSE)}

\cppcodefile[label=code:SseHeatCapacityAnalyzer]{Classes/Analyzer/SseHeatCapacityAnalyzer.cpp}

%=======================================

\newpage
\thispagestyle{empty}
\cleardoublepage
\let\listfigurenamePARENT\listfigurename
\renewcommand{\listfigurename}{\addcontentsline{toc}{chapter}{\listfigurenamePARENT}\listfigurenamePARENT}
\begingroup
\renewcommand*{\addvspace}[1]{}
\listoffigures
\endgroup

\renewcommand{\lstlistlistingname}{\addcontentsline{toc}{chapter}{Quellcodeverzeichnis}Quellcodeverzeichnis}
\lstlistoflistings

\let\bibnamePARENT\bibname
\renewcommand{\bibname}{\addcontentsline{toc}{chapter}{\bibnamePARENT}\bibnamePARENT}
\bibliographystyle{alpha}
\bibliography{Literatur}

\chapter*{Erkl�hrung zur Selbstst�ndigkeit}
\addcontentsline{toc}{chapter}{Erkl�hrung zur Selbstst�ndigkeit}

Hiermit versichere ich, dass ich diese Arbeit selbstst�ndig verfasst und keine anderen als die angegebenen Quellen und Hilfsmittel benutzt habe.

\vspace{10mm}
\begin{tabular}{@{}p{40mm}@{\hspace{1mm}}l@{\hspace{1mm}}p{30mm}@{\hspace{6mm}}p{50mm}}
\hrule&,&\hrule&\hrule\\[-3mm]
Ort&&Datum&Unterschrift\\
\end{tabular}

\end{document}